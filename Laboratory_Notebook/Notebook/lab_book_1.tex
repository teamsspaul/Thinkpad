%%%%%%%%%%%%%%%%%%%%%%%%%%%%%%%%%%%%%%%%%
% Daily Laboratory Book
% LaTeX Template
%
% This template has been downloaded from:
% http://www.latextemplates.com
%
% Original author:
% Frank Kuster (http://www.ctan.org/tex-archive/macros/latex/contrib/labbook/)
%
% Important note:
% This template requires the labbook.cls file to be in the same directory as the
% .tex file. The labbook.cls file provides the necessary structure to create the
% lab book.
%
% The \lipsum[#] commands throughout this template generate dummy text
% to fill the template out. These commands should all be removed when 
% writing lab book content.
%
% HOW TO USE THIS TEMPLATE 
% Each day in the lab consists of three main things:
%
% 1. LABDAY: The first thing to put is the \labday{} command with a date in 
% curly brackets, this will make a new page and put the date in big letters 
% at the top.
%
% 2. EXPERIMENT: Next you need to specify what experiment(s) you are 
% working on with an \experiment{} command with the experiment shorthand 
% in the curly brackets. The experiment shorthand is defined in the 
% 'DEFINITION OF EXPERIMENTS' section below, this means you can 
% say \experiment{pcr} and the actual text written to the PDF will be what 
% you set the 'pcr' experiment to be. If the experiment is a one off, you can 
% just write it in the bracket without creating a shorthand. Note: if you don't 
% want to have an experiment, just leave this out and it won't be printed.
%
% 3. CONTENT: Following the experiment is the content, i.e. what progress 
% you made on the experiment that day.
%
%%%%%%%%%%%%%%%%%%%%%%%%%%%%%%%%%%%%%%%%%

%---------------------------------------------------------------------
%	PACKAGES AND OTHER DOCUMENT CONFIGURATIONS
%----------------------------------------------------------------------

\documentclass[idxtotoc,hyperref,openany]{labbook} % 'openany' here removes the gap page between days, erase it to restore this gap; 'oneside' can also be added to remove the shift that odd pages have to the right for easier reading

\usepackage[ 
  backref=page,
  pdfpagelabels=true,
  plainpages=false,
  colorlinks=true,
  bookmarks=true,
  pdfview=FitB]{hyperref} % Required for the hyperlinks within the PDF
  
\usepackage{booktabs} % Required for the top and bottom rules in the table
\usepackage{float} % Required for specifying the exact location of a figure or table
\usepackage{graphicx} % Required for including images
\usepackage{lipsum} % Used for inserting dummy 'Lorem ipsum' text into the template
\usepackage{mathrsfs}
\usepackage{amsmath}

\newcommand{\HRule}{\rule{\linewidth}{0.5mm}} % Command to make the lines in the title page
\setlength\parindent{0pt} % Removes all indentation from paragraphs

\usepackage{enumitem,amssymb}
\newlist{todolist}{itemize}{2}
\setlist[todolist]{label=$\square$}
\usepackage{pifont}
\newcommand{\cmark}{\ding{51}}%
\newcommand{\xmark}{\ding{55}}%
\newcommand{\done}{\rlap{$\square$}{\raisebox{2pt}{\large\hspace{1pt}\cmark}}%
  \hspace{-2.5pt}}
\newcommand{\wontfix}{\rlap{$\square$}{\large\hspace{1pt}\xmark}}
\newcommand{\tss}{\textsuperscript}
\newcommand{\tsbs}{\textsubscript}

%---------------------------------------------------------------------
%	DEFINITION OF EXPERIMENTS
%---------------------------------------------------------------------

\newexperiment{Notes}{Experiment Notes}
\newexperiment{Stock}{Stock creation and count}
\newexperiment{table}{This shows a sample table}
%\newexperiment{shorthand}{Description of the experiment}

%----------------------------------------------------------------------
\begin{document}

%----------------------------------------------------------------------
%	TITLE PAGE
%-----------------------------------------------------------------------

\frontmatter % Use Roman numerals for page numbers
\title{
\begin{center}
\HRule \\[0.4cm]
{\Huge \bfseries Laboratory Journal }\\[0.4cm] % Degree
\HRule \\[1.5cm]
\end{center}
}
\author{\Huge Paul Mendoza \\ \\ \LARGE paul.m.mendoza@gmail.com \\[2cm]}
\date{Beginning 6 October 2016} 
\maketitle

\tableofcontents

\mainmatter 

%------------------------------------------------------------------------
%	LAB BOOK CONTENTS
%------------------------------------------------------------------------




%------------------------------------------------------------------------
%	LAB BOOK day
%-----------------------------------------------------------------------

\labday{Thursday, 6 October 2016\\ 8:30am - 11:00 am\\ 1:30pm - 5:30pm}

\experiment{Notes}

\begin{itemize}
  \item{Project Number: 504370-0001}
\end{itemize}


\experiment{Stock} % 

\begin{itemize}
\item{Get stock solution from Troy room 18A, store near rad waste}
\item{Grab 1000$\mu$l pipett from glovebox}
\item{Decontaminate with radic - dump waste into glass aq rad outside glove box}
\item{Practice pipetting 500$\mu$l to glass vial - setting 503 $\mu$l
  gives 500 $\mu$l}
\item{Class/lunch Break}
\item{Get alpha detector from Dr. Marianno}
\item{Set up laboratory notebook}
\item{Calculation}
  To do calculation to determine the volumes needed for a final
  concentration of a particular volume, knowing the initial
  concentrations
  \begin{align*}
    V_2&=\frac{b_2-\frac{M_1b_1}{A}}{M_2-\frac{M_1}{A}}\\
    V_1&=\frac{b-BV_2}{A}
  \end{align*}
  Where:
  \begin{align*}
    A&=(1-wt\%_1)\rho_1\\
    B&=(1-wt\%_2)\rho_2\\
    b_1&=(1-wt\%_3)V_3\rho_3\\
    b_2&=M_3V_3
  \end{align*}
  With known Molarity and volume of a solution
  how much, and of what concentration
  do we need to combine with a second solution
  to get a final solution of known concentration
  and volume?
  \begin{align*}
    B&=(1-wt\%_3)V_3\rho_3-(1-wt\%_1)V_1\rho_!\\
    A&=M_3V_3-M_1V_1\\
    C&=\frac{B}{A}=\frac{(1-wt\%_2)\rho_2}{M_2}
  \end{align*}
  Need iterative solution, choose:
  \begin{align*}
    M_2&=\frac{M_3V_3-M_1V_1}{V_3-V_1}\\
    V_2&=V_3-V_1
  \end{align*}
  Use to determine molality $\rightarrow$ $wt\%_2$ $\rightarrow$
  $\rho_2$. Then compare to $C$, iterate around the solution to
  find answer so that $C=\frac{(1-wt\%_2)\rho_2)}{M_2}$.
\end{itemize}



%------------------------------------------------------------------------
%	LAB BOOK day
%-----------------------------------------------------------------------

\labday{Friday, 7 October 2016\\ 9:00am - 12:00 am\\ 1:00pm - 5:00pm}

\experiment{Stock} % 

\begin{todolist}
\item[\done]{Program calculation for creation of stock}
\item{-}
\end{todolist}
\begin{center}
    0.149+/-0.011 ml of 15.43+/-0.06 M HNO\tsbs{3} solution\\
    +\\
    1.91+/-0.08  ml of 0.0+/-0 M solution\\
    = \\
    2.048+/-0.026 ml of 1.12+/-0.08 M HNO\tsbs{3}
  solution $\boxed{\rightarrow Stock}$ (glass container)
\end{center}
\begin{todolist}
\item{-}
\end{todolist}
\begin{center}
Combine 0.500+/-0.005 ml of 15.43+/-0.06 M HNO\tsbs{3}
solution $\boxed{closet}$\\
+\\
2.048+/-0.026 ml of 1.12+/-0.08 M HNO\tsbs{3} solution
$\boxed{Stock}$
\\
=\\
2.500+/-0.025 ml of 4.00+/-0.05 M HNO\tsbs{3} solution.
$\boxed{\rightarrow Stock}$
\end{center}

\begin{todolist}
\item{Put Source back in rad closet}
\item{Remove 0.3 ml from $\boxed{ Stock}$ to $\boxed{1}$ count
      on HPGe 30 minutes}
\end{todolist}



%----------------------------------------------------------------------
%	Examples
%-----------------------------------------------------------------------
%---------------------------------------
% Blank template to use for new days:
%---------------------------------------
%\labday{Day, Date Month Year}
%\experiment{}
%Text

%---------------------------------------
% To do list
%---------------------------------------
%% \begin{todolist}
%% \item[\wontfix]{profit}
%% \item[\done]{This is done}
%% \item{This is not done}
%% \end{todolist}

%---------------------------------------
% Vial Step
%---------------------------------------
%% \begin{todolist}
%% \item{-}
%% \end{todolist}
%% \begin{center}
%% Vial 1\\
%% +\\
%% Vial 2\\
%% =\\
%% Final Vial
%% \end{center}

\labday{Example} % We don't want a date here so we make the labday blank

\begin{center}
\HRule \\[0.4cm]
{\huge \textbf{Examples}}\\[0.4cm] % Heading
\HRule \\[1.5cm]
\end{center}

%-----------------------------------------------------------------------
%	Formulae
%-----------------------------------------------------------------------
\huge \textbf{Formulae} \\ \\

\normalsize \textbf{Formula 1 - Pythagorean theorem}\\ \\
$a^2 + b^2 = c^2$\\ \\

%--------------------------------------------------------------
%	Citation
%--------------------------------------------------------------

Citation test \cite{Tatro2013}.

%--------------------------------------------------------------
%	Figure
%--------------------------------------------------------------

\begin{figure}[H] % Example of including images
\begin{center}
\includegraphics[width=0.5\linewidth]{Figures/example_figure}
\end{center}
\caption{Example figure.}
\label{fig:example_figure}
\end{figure}

%--------------------------------------------------------------
%	Table
%--------------------------------------------------------------

\experiment{table}

\begin{table}[H]
\begin{tabular}{l l l}
\toprule
\textbf{Groups} & \textbf{Treatment X} & \textbf{Treatment Y} \\
\toprule
1 & 0.2 & 0.8\\
2 & 0.17 & 0.7\\
3 & 0.24 & 0.75\\
4 & 0.68 & 0.3\\
\bottomrule
\end{tabular}
\caption{The effects of treatments X and Y on the four groups studied.}
\label{tab:treatments_xy}
\end{table}

Table \ref{tab:treatments_xy} shows that groups 1-3 reacted similarly to the two treatments but group 4 showed a reversed reaction.

%------------------------------------------------------------
%	Bibliography
%------------------------------------------------------------
\bibliography{references} 
\bibliographystyle{plain} 

\end{document}

