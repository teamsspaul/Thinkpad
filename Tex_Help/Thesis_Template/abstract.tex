% +--------------------------------------------------------------------+
% | Abstract Page
% +--------------------------------------------------------------------+

\pagestyle{empty}
%\vspace{1cm}
\setlength{\baselineskip}{0.8cm}

%\indent

% +--------------------------------------------------------------------+
% | Enter the text of your abstract below, maximum of 500 words.
% +--------------------------------------------------------------------+

Enter the text of your abstract in the abstract.tex file.  Be sure
to delete the text below before you submit your ETDR.

This template uses a separate file for each section of your ETDR:
title page, abstract, preface, chapters, reference, etc.  This
makes it easier to organize and work with a lengthy document.  The
template is configured with page margins required by the Graduate
School and will automatically create a table of contents, lists of
tables and figures, and PDF bookmarks.

The file etdrtemplate.tex is the "master" file for the ETDR
template.  This is the file you need to process with PDFLaTeX in
order to produce a PDF version of your ETDR.  See the comments in
the etdrtemplate.tex and other files for details on using the
template.  You are not required to use the template, but it can save
time and effort in making sure your ETDR meets the Graduate School
formatting requirements.

Although the template gives you a foundation for creating your
ETDR, you will need a working knowledge of LaTeX in order to
produce a final document.  You should be familiar with LaTeX
commands for formatting text, equations, tables, and other
elements you will need to include in your ETDR.
